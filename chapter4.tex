\chapter{Empfehlungssysteme}\label{chap:rec}

Empfehlungssysteme sind eine Anwendungsart der Singulärwertzerlegung, mit der bereits die meisten Menschen in Kontakt gekommen sind.
Seien es Filmempfehlungen bei Netflix oder Produktempfehlungen bei Amazon, die Wahrscheinlichkeit ist groß, dass diese mithilfe einer Abwandlung der SVD generiert werden.
In diesem Kapitel wird zunächst, ähnlich wie im vorherigen Kapitel, die Grundidee von Empfehlungssystemen mithilfe eines intuitiven Ansatzes veranschaulicht.
Anschließend wird diese Idee mathematisch formalisiert und vertieft.
Um das Kapitel abzuschließen, wird mithilfe von \texttt{Python} ein eigenes (simples) Empfehlungssystem für Filme programmiert, basierend auf tatsächlichen Bewertungen aus einer Datenbank.

\section{Grundidee}

Für die grundlegende Idee von Empfehlungssystemen verweilen wir beim Beispiel der Filmempfehlungen.
Die Ausgangslage für ein solches Empfehlungssystem ist in \zcref{tab:rec:usit} veranschaulicht.
\begin{table}[tb]
    \centering
    \caption{Nutzer-Item-Matrix}\label{tab:rec:usit}
    \input{tables/rec_usit.tex}
\end{table}

Gegeben ist eine Nutzer-Item-Matrix, in der jede Zeile einen Nutzer und jede Spalte einen Film repräsentiert, wobei die einzelnen Einträge die abgegebenen Bewertungen der Nutzer für den jeweiligen Film darstellen.
Das Ziel des Empfehlungssystems ist, basierend auf den vorhandenen Bewertungen möglichst präzise Empfehlungen zu generieren.
Dies kann zum einen durch eine Approximation und anschließende Sortierung der fehlenden Bewertungen erfolgen.
Eine andere Möglichkeit besteht darin, die Ähnlichkeit verschiedener Nutzer oder Filme zu analysieren.
Für beide Vorgehensweisen wird die Annahme getroffen, dass die Bewertungen nicht unabhängig erfolgen, sondern einer bestimmten Struktur folgen.
Es wird also angenommen, dass es zugrunde liegende Muster gibt, nach denen Nutzer mit ähnlichen Präferenzen auch tendenziell ähnliche Bewertungen vergeben.
Ein Beispiel dafür wäre, dass Nutzer mit einer Vorliebe für Horrorfilme diese potenziell höher bewerten als andere Nutzer.
Diese Muster werden als \emph{latente Merkmale} bezeichnet.

Im Folgenden werden die Merkmale genutzt, indem die Nutzer-Item-Matrix \(R\) als Produkt zweier Matrizen dargestellt wird:
einer Nutzer-Matrix \(U\), in der die Nutzer durch die latenten Merkmale beschrieben werden, und einer Item-Matrix \(V\) mit der Beschreibung der Filme durch die Merkmale. 
Dieses Konzept wird in \zcref{fig:rec:twomat} verdeutlicht mit den latenten Merkmalen \(X_{1}\) und \(X_{2}\).  
\begin{figure}[tb]
    \begin{equation*}
        \begin{bNiceMatrix}%
            [
                first-row,
                first-col,
                code-for-first-row = \scriptstyle \Alph{jCol},
                code-for-first-col = \scriptstyle \arabic{iRow},
                margin = 2pt,
            ]
            &&&&\\
            && 4 & 2 & 0 \\
            &0 & 2 & 3 & 5 \\
            &1 & 2 & \cellcolor{red!15} & \\
            && 4 & 3 & 3 \\
            &4 & 2 & 1 & 1 \\
            &5 &&& 2 \\
            \CodeAfter
            \UnderBrace[yshift=3pt]{6-1}{6-4}{R}
        \end{bNiceMatrix}
        \quad
        \approx
        \quad
        \begin{bNiceMatrix}%
            [
                first-row,
                first-col,
                code-for-first-row = \scriptstyle,
                code-for-first-col = \scriptstyle \arabic{iRow},
                margin = 2pt,
            ]
            \CodeBefore
            \rectanglecolor{red!15}{3-1}{3-2}
            \Body
            & X_{1} & X_{2} \\
            & ? & ?  \\
            & ? & ?  \\
            & ? & ?  \\
            & ? & ?  \\
            & ? & ?  \\
            & ? & ?  \\
            \CodeAfter
            \UnderBrace[yshift=3pt]{6-1}{6-2}{U^{'}}
        \end{bNiceMatrix}
        \quad
        \times
        \quad
        \begin{bNiceMatrix}%
            [
                first-row,
                last-col = 5,
                code-for-first-row = \scriptstyle \Alph{jCol},
                code-for-last-col = \scriptstyle,
                margin = 2pt,
            ]
            \CodeBefore
            \columncolor{red!15}{3}
            \Body
            &&& \\
            ? & ? & ? & ? & X_{1} \\
            ? & ? & ? & ? & X_{2} \\
            \CodeAfter
            \UnderBrace[yshift=6pt]{2-1}{2-4}{V^{'}}
        \end{bNiceMatrix}
    \end{equation*}
    \vspace{4pt}
    \caption{Nutzer-Matrix und Item-Matrix}\label{fig:rec:twomat}
\end{figure}

Fehlende Bewertungen können damit als Skalarprodukt der jeweiligen Vektoren approximiert werden, wie in \zcref{fig:rec:twomat} farblich hervorgehoben ist.
Zudem kann die Ähnlichkeit zwischen Nutzern oder Filmen analysiert werden, indem diese als Punkte im Raum der latenten Merkmale dargestellt werden.
Es bleibt damit die Frage, wie die Matrizen \(U^{'}\) und \(V^{'}\) berechnet werden können.

\section{Mathematische Herleitung}

Mit Blick auf den Rahmen dieser Arbeit liegt es nahe, dass diese Frage mithilfe der Singulärwertzerlegung beantwortet werden kann.
Sei also 
\begin{equation*}
    R \approx U_{k} \Sigma_{k} V_{k}^{T}
\end{equation*}
für \(k < \rg(R)\) die trunkierte SVD wie in \zcref{df:trunsvd}. 
Wir wollen zeigen, dass eine Darstellungsform der gesuchten Matrizen durch
\begin{equation}
    U^{'} = U_{k} \sqrt{\Sigma_{k}}, \quad V^{'} = \sqrt{\Sigma_{k}}V^{T}_{k} \label{eq:svdrec}
\end{equation}
gegeben ist.
Ein intuitives Verständnis für diese Verbindung zwischen der SVD und der Nutzer-, bzw.\ Item-Matrix bieten bereits bewiesene Aussagen im vorherigen Teil der Arbeit.
Da sich die Nutzer als Linearkombination der Filme darstellen lassen, befinden sie sich im Spaltenraum von \(R\). 
Nach \zcref{cor:svd} bilden die Spalten von \(U\) eine Basis für diesen Raum, was sich auch in \zcref{fig:rec:twomat} widerspiegelt:
Jeder Nutzer kann ebenfalls als Linearkombination der latenten Merkmale ausgedrückt werden.
Eine analoge Argumentation gilt für die Filme über den Zeilenraum.

Da durch \(\Sigma_{k}\) nur eine Streckung erfolgt und die Basen damit erhalten bleiben, erhalten wir die gewünschte Darstellung in \eqref{eq:svdrec}.
Diese bietet nach \zcref{th:eckyou} sogar die beste Rang-\(k\)-Approximation für \(R\).   


