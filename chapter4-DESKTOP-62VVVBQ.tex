\chapter{Empfehlungssysteme}

Empfehlungssysteme sind eine Anwendungsart der Singulärwertzerlegung, mit der bereits die meisten Menschen in Kontakt gekommen sind.
Seien es Filmempfehlungen bei Netflix oder Produktempfehlungen bei Amazon, die Wahrscheinlichkeit ist groß, dass diese mithilfe einer Abwandlung der SVD generiert werden.
In diesem Kapitel wird zunächst, ähnlich wie im vorherigen Kapitel, die Grundidee von Empfehlungssystem mithilfe eines intuitiven Ansatzes veranschaulicht.
Anschließend wird diese Idee mathematisch formalisiert und vertieft.
Um das Kapitel abzuschließen, wird mithilfe von \texttt{Python} ein eigenes (simples) Empfehlungssystem für Filme programmiert, basierend auf tatsächlichen Bewertungen aus einer Datenbank.

\section{Intuition}

Für die Intuition verweilen wir beim Beispiel der Filmempfehlungen.
Die Ausgangslage für so ein Empfehlungssystem ist in \zcref{tab:rec:usit} veranschaulicht.
\begin{table}[tb]
    \centering
    \caption{Nutzer-Item-Matrix}\label{tab:rec:usit}
    \input{tables/rec_usit.tex}
\end{table}

Gegeben sei eine Nutzer-Item-Matrix, in der jede Zeile einen Nutzer und jede Spalte einen Film repräsentiert, wobei die einzelnen Einträge die abgegebenen Bewertungen der Nutzer für den jeweiligen Film darstellen.
Das Ziel des Empfehlungssystems besteht darin, die fehlenden Bewertungen so präzise wie möglich zu approximieren, um darauf basierend Empfehlungen generieren zu können.
Hierfür wird die Annahme getroffen, dass die Bewertungen nicht unabhängig erfolgen, sondern einer bestimmten Struktur folgen.
Es wird also angenommen, dass es zugrunde liegende Muster gibt, nach denen Nutzer mit ähnlichen Präferenzen auch tendenziell ähnliche Bewertungen vergeben.
Ein Beispiel dafür wäre, dass Nutzer mit einer Vorliebe für Horrorfilme diese potenziell höher bewerten als andere Nutzer.

Diese Muster werden als \emph{latente Merkmale} bezeichnet.
Zur Approximation der fehlenden Bewertungen wird die Nutzer-Item-Matrix als Produkt zweier Matrizen dargestellt:
einer Nutzer-Matrix, in der die Nutzer durch die latenten Merkmale beschrieben werden, und einer Item-Matrix mit der Beschreibung der Filme durch die Merkmale. 
Dieses Konzept wird in \zcref{fig:rec:twomat} verdeutlicht mit den latenten Merkmalen \(X_{1}\) und \(X_{2}\).  
\begin{figure}[tb]
    \begin{equation*}
        \begin{bNiceMatrix}%
            [
                first-row,
                first-col,
                code-for-first-row = \scriptstyle \Alph{jCol},
                code-for-first-col = \scriptstyle \arabic{iRow}
            ]
            &&&&\\
            && 4 & 2 & 0 \\
            &0 & 2 & 3 & 5 \\
            &1 & 2 & \cellcolor{red!15} & \\
            && 4 & 3 & 3 \\
            &4 & 2 & 1 & 1 \\
            &5 &&& 2 \\
        \end{bNiceMatrix}
        \quad
        \approx
        \quad
        \begin{bNiceMatrix}%
            [
                first-row,
                first-col,
                code-for-first-row = \scriptstyle,
                code-for-first-col = \scriptstyle \arabic{iRow}
            ]
            \CodeBefore
            \rowcolor{red!15}{3}
            \Body
            & X_{1} & X_{2} \\
            & ? & ?  \\
            & ? & ?  \\
            & ? & ?  \\
            & ? & ?  \\
            & ? & ?  \\
            & ? & ?  \\
        \end{bNiceMatrix}
        \quad
        \times
        \quad
        \begin{bNiceMatrix}%
            [
                first-row,
                last-col = 5,
                code-for-first-row = \scriptstyle \Alph{jCol},
                code-for-last-col = \scriptstyle
            ]
            \CodeBefore
            \columncolor{red!15}{3}
            \Body
            &&& \\
            ? & ? & ? & ? & X_{1} \\
            ? & ? & ? & ? & X_{2} \\
        \end{bNiceMatrix}
    \end{equation*}
    \caption{Nutzer-Matrix und Item-Matrix}\label{fig:rec:twomat}
\end{figure}

Für eine Approximation des hervorgehobenen fehlenden Wertes muss dann nur das Skalarprodukt aus den markierten Vektoren gebildet werden.